
% Preamble
\documentclass[a4paper]{article} % Uses article class in A4 format

%----------------------------------------------------------------------------------------
%	FORMATTING
%----------------------------------------------------------------------------------------

\addtolength{\hoffset}{-2.25cm}
\addtolength{\textwidth}{4.5cm}
\addtolength{\voffset}{-3.25cm}
\addtolength{\textheight}{5cm}
\setlength{\parskip}{0pt}
\setlength{\parindent}{0in}


%----------------------------------------------------------------------------------------
%	PACKAGES AND OTHER DOCUMENT CONFIGURATIONS
%----------------------------------------------------------------------------------------

\usepackage[utf8]{inputenc} % Use UTF-8 encoding

\usepackage[english]{babel} % Language hyphenation and typographical rules

\usepackage{amsthm, amsmath, amssymb, bm} % Mathematical typesetting
\usepackage{float} % Improved interface for floating objects
\usepackage{graphicx, multicol} % Enhanced support for graphics
\usepackage{subcaption}
\usepackage{xcolor} % Driver-independent color extensions
\usepackage{listings}
\usepackage{caption}

%\usepackage{csquotes} % Context sensitive quotation facilities
\usepackage[yyyymmdd]{datetime} % Uses YEAR-MONTH-DAY format for dates
\renewcommand{\dateseparator}{-} % Sets dateseparator to '-'

\usepackage{fancyhdr}
\usepackage{amsmath}
\usepackage{bm} % Headers and footers
\pagestyle{fancy} % All pages have headers and footers
\fancyhead{}\renewcommand{\headrulewidth}{0pt} % Blank out the default header
\fancyfoot[L]{} % Custom footer text
\fancyfoot[C]{} % Custom footer text
\fancyfoot[R]{\thepage} % Custom footer text

%\newcommand{\note}[1]{\marginpar{\scriptsize \textcolor{red}{#1}}} % Enables comments in red on margin
\DeclareMathOperator*{\argmax}{argmax}
\DeclareMathOperator*{\argmin}{argmin}
%----------------------------------------------------------------------------------------

% Document
\begin{document}
%----------------------------------------------------------------------------------------

%	TITLE SECTION
    \title{Title} % Article title
    \fancyhead[C]{}
    \hrule \medskip % Upper rule
    \begin{minipage}{0.295\textwidth} % Left side of title section
        \raggedright
        TTK4255\\ % Your lecture or course
        \footnotesize % Authors text size
        \hfill\\
        Fabian Höldin % Your name, your matriculation number
    \end{minipage}
    \begin{minipage}{0.4\textwidth} % Center of title section
        \centering
        \large % Title text size
        Robotic Vision\\ % Assignment title and number
        \normalsize % Subtitle text size
        Homework 5: Two-view geometry\\ % Assignment subtitle
    \end{minipage}
    \begin{minipage}{0.295\textwidth} % Right side of title section
        \raggedleft
        \today\\ % Date
        \footnotesize % Email text size
        \hfill\\
        % Your email
    \end{minipage}
    \medskip\hrule % Lower rule
%----------------------------------------------------------------------------------------
%	ARTICLE CONTENTS
%----------------------------------------------------------------------------------------

\section{The epipolar constraint}
    \subsection*{Task 1.1}
    Multiplying $\bm{\tilde{l}}$ with an arbitrary scalar $s$ results in the following equation:
    \begin{align*}
        \bm{\tilde{x}}^T * s * \bm{\tilde{l}} &= 0, \\
        x* s \cos{\Theta} + y* s \sin{\Theta} &= s* \rho \quad  | \div s , \\
        x \cos{\Theta} + y \sin{\Theta} &= \rho
    \end{align*}
    As you can see, scalar factor $s$ has no influence on the equation.

    An arbitrary homogeneous line $\bm{\tilde{l}} = (a, b, c)$ by division through $c$
    

    \subsection*{Task 1.2}



\end{document}